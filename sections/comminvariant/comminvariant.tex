% !TEX root = ../../mat830_mckay.tex
\newpage
\section{Commutative Algebra of Invariant Rings}
\subsection{Noether's Theorem}

There are two standard approaches to this topic: a graded ring approach and a local ring approach. We shall take the former. Let $k$ be a field and $W$ a $d$-dimensional $k$-vector space. Furthermore, let $G \subseteq \GL_n(W) \cong \GL_n(k)$ be a finite group with $|G| \in k^\times$. Define $S=k[W] \cong k[x_1,\ldots,x_d]$ and let $R= S^G$ be its invariant ring. In particular, $R \subseteq S$ is a subalgebra. Now invariant rings are `nicer' than a typical randomly chosen subring of a polynomial ring, by which we mean that there is a `nice' operator on the invariant subring, namely the Reynold's operator $\rho: S \to R$ given by
		\[
		\rho(s)= \dfrac{1}{|G|} \sum_{g \in S} g_S.
		\]
The Reynold's operator is an $R$-linear split surjection, so $S \cong R \oplus \ker \rho$ as $R$-modules. Furthermore, when working with the ring of invariants, one works with symmetric polynomials, which are inherently `nicer' than typical functions. The goal of this section will be to prove the following theorem of Noether:


% \noether28
\begin{restatable}[Noether, 1928]{thm}{noether28} \label{thm:noether28}
The ring $R$ is a noetherian integrally closed domain of Krull dimension $d= \dim_k W$. 
\end{restatable}


\begin{rem}
None of these properties of Noether's Theorem hold for arbitrary subrings of polynomial rings. 
\end{rem}


Recall that the Krull dimension of a $k$-algebra $A$ is the maximal number of algebraically independent elements over $k$. This definition is equivalent to the transcendence degree of the quotient field $Q(A)$ over $k$. If $A$ is the $\C$-algebra given by the coordinate ring of a collection of polynomials, xthen the Krull dimension is the topological dimension of the vanishing set corresponding to $A$; that is, if $A= k[y_1,\ldots,y_m]/(f_1,\ldots,f_r)$, then $\dim A= \dim V(f_1,\ldots,f_r) \subseteq \C^m$. We first show that a polynomial ring is an integral extension of its invariants ring.
 
 
 \begin{lem}
 Let $S=[k_1,\ldots,x_d]$ and $R=S^G$ its ring of invariants. The extension $R \hookrightarrow S$ is integral, i.e. every element of $S$ is a root of a monic polynomial with coefficients in $R$. 
 \end{lem}

\pf For $s \in S$, define $f_s(t):= \prod_{g \in G} (t-gs)$. Then $f_s(t)$ is monic in $t$ and has $s$ as a root since $t-s$ is a factor of $f_s(t)$. The action of $G$ permutes the factors of $f_s(t)$, so it fixes the coefficients of $f_s(t)$. Therefore, $f_s(t) \in R[x]$. \qed \\


This already shows that $\dim R= \dim S= d$ since integral extensions satisfy Going-Up and Going-Down, and thus preserve dimension. We know also that if $A \hookrightarrow B$ is an integral extension of rings and $B$ is finitely generated as an $A$-algebra, then $B$ is finitely generated as an $A$-module (see Atiyah-MacDonald~\cite[Ch.~5]{atiyahmac}). It then follows that


\begin{cor}
Let $S=[k_1,\ldots,x_d]$ and $R=S^G$ its ring of invariants. Then $S$ is a finitely generated $R$-module of rank $|G|$.
\end{cor}

\pf It is enough to show that $S$ is a finitely generated $R$-algebra. Now $S$ is generated over $k$ by $x_1,\ldots,x_d$, so that it is generated over $R$ by $x_1, \ldots, x_d$ as well so that $S$ is a finitely generated $R$-module. It remains to show that $S$ has rank $|G|$. Recall that the rank of a module $M$ over a domain $A$ is $\dim_{Q(A)} M \otimes_A Q(A)$. So we need to compute $\dim_{Q(R)}(S \otimes_R Q(R))$.  But $S \otimes_R Q(R)$ is just $Q(S)$ by integrality. Then any $s \in S$ satisfies an equation $s^p + r_1s^{p-1}+\cdots+r_{p-1}s+r_p=0$, where $r_i \in R$ and $r_p \neq 0$. Then $s(s^{p-1}+r_1s^{p-2}+\cdots+r_{p-1})= -r_p$. Then
	\[
	\dfrac{1}{s}= -\dfrac{1}{r_p} \left( r_1 s^{p-2} + \cdots + r_{p-1} \right).
	\]
If you invert everything in $R$, then you have effectively inverted everything in $s$ by the above. So we need $\dim_{Q(R)} Q(S)$. This a Galois field extension, $Q(S)^G= Q(R)$ so it has degree $|G|$. \qed \\


We now prove that $R$ is noetherian: \\

We know that $S$ is integral over $R$, so that each $x_i$ is integral over $R$. Let $f_1(t), \ldots, f_d(t)$ be monic polynomials with $f_i(x_i)=0$. Let $B \subseteq R$ be the $k$-subalgebra generated by all the coefficients of $f_1,\ldots,f_d$. Then $B$ is finitely generated as a $k$-algebra so that it is noetherian by the Hilbert Basis Theorem. $S$ is integral over $B$, and finitely generated as a $B$-algebra, so finitely generated as a $B$-module. Now $B$ is noetherian and $S$ is a finitely generated $B$-module, every $B$-submodule of $S$ is a finitely generated module. In particular, $R$ must be a finitely generated $B$-module and hence is a noetherian module. Ideals in $R$ are $B$-submodules of $R$ and hence satisfy the ascending chain condition. Hence, $R$ is noetherian. \qed \\



We now prove that $R$ is integrally closed in its quotient field (normal): \\


We know that $Q(S)^= Q(R)$. 


% Field diagram
% 		     Q(S)
%                /        \
% 		S	     \
% 		  \          Q(R)
% 	                R/		


If $x=a/b \in Q(R)$ is integral over $R$, then in particular $x \in Q(S)$ and is integral over $S$. Polynomial rings are integrally closed in their quotient field. So $x \in S$. Hence, $x \in S \cap Q(R)= S \cap Q(S)^G= S^G= R$. When is $R$ a UFD? Always? Never? In between? We know that $S \cong k[x_1,\ldots,x_d]$ is a UFD but $k[u,v]^{C_2}= k[u^2,uv,v^2] \cong k[a,b,c]/(ac-b^2)$ is not a UFD since $u^2 \cdot v^2= (uv)^2$. If we assume that $G$ has no nontrivial linear characters, i.e. no group homomorphisms $G \to k^\times$, then the invariant ring is a UFD. 

\pf Let $r \in R$, then $r \in S$, so we can factor $r= q_1^{a_1} \cdots q_m^{a_n}$, where each $q_i$ is irreducible in $S$ are pairwise non-associate. The $q_i$'s are not necessarily permuted by $G$, but the ideals $(q_1),\ldots,(q_m) \subseteq S$ are. Let $O_1,\ldots,O_e$ be the orbits of the ideals $(q_1),\ldots,(q_m)$ under this action and set
	\[
	Q_j = \prod_{(q_i) \in O_j} q_i^{a_i}
	\]
for $j=1,\ldots,e$. The ideal $(Q_j)$ is stable under the action of $G$, so in particular for any $g \in G$, $gQ_j \in (Q_j)$ so $gQ_j$ is a multiple of $Q_j$. For degree reasons, $gQ_j= \lambda_g Q_j$ for some $\lambda_g \in k^\times$. The function $\lambda: G \to k^\times$ given by $g \mapsto \lambda_g$ (we get one such function for each orbit) and it is a group homomorphism. By assumption, $\lambda$ must be trivial, i.e. $gQ_j= Q_j$ for every $g \in G$ and $j=1,\ldots,e$. Then $Q_j \in R$ for every $j$ and $r= Q_1Q_2 \cdots Q_e$ is a factorization in $R$. Each $Q_j$ is irreducible because it is a product over a single orbit. It is routine, although tedious, to check uniqueness. \qed \\


\begin{ex}
$R= S^G$ is a UFD in the following cases (this is not a complete list, just examples): if $G$ is a nonabelian simple group, e.g. $A_n$ for $n \geq 5$ note that $A_5$ is the $\cI \cong A_5$ group, then we see that the $E_8$ singularity defined by $x^2+y^3+z^5=0$ is a UFD. Or $G= [G,G]$, i.e. every element is a commutator or a product of commutators (in $k^\times$ they are forced to be abelian so go to 1), such groups are called perfect. OR $k= \Q$ and $G$ has odd order (so then does every element) and every image of every element must then be of odd order, which there are none. One can generalize this to $|k|= p^e$ and $\gcd(|G|,|k^\times|=q-1)$. 
\end{ex}


Having proved Noether's Theorem, our next goal is the Hochster-Eagon Theorem that $R= S^G$, i.e. invariant rings are Cohen-Macaulay. We work with graded rings (recall that $k[x_1,\ldots,x_n]= \bigoplus_p k[x_1,\ldots,x_n]_p$, the $p$th graded piece consisting of homogeneous polynomials of degree $p$ and our group actions preserve degree so $k[x_1,\ldots,x_n]^= \bigoplus_p k[x_1,\ldots,x_n]_p^G$. Let $A= \bigoplus_{i=1}^\infty A_i= A_0 \oplus A_1 \oplus \cdots$ be a graded ring with $A_0= k$ a field. Sometimes [Bruns-Herzog] such rings are called *local since they have a unique homogeneous maximal ideal $\fm= A_1 \oplus A_2 \oplus \cdots$. 


\begin{dfn}[Homogenous System of Parameters]
A homogeneous system of parameters for $A$ is a sequence of elements $a_1,\ldots,a_t$ such that $A$ is a finitely generated module over the (graded subring) $k[a_1,\ldots,a_t]$. 
\end{dfn}






