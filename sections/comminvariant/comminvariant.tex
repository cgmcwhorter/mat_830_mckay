% !TEX root = ../../mat830_mckay.tex
\newpage
\section{Commutative Algebra of Invariant Rings}
\subsection{Noether's Theorem}

There are two standard approaches to this topic: a graded ring approach and a local ring approach. We shall take the former. Let $k$ be a field and $W$ a $d$-dimensional $k$-vector space. Furthermore, let $G \subseteq \GL_n(W) \cong \GL_n(k)$ be a finite group with $|G| \in k^\times$. Define $S=k[W] \cong k[x_1,\ldots,x_d]$ and let $R= S^G$ be its invariant ring. In particular, $R \subseteq S$ is a subalgebra. Now invariant rings are `nicer' than a typical randomly chosen subring of a polynomial ring, by which we mean that there is a `nice' operator on the invariant subring, namely the Reynold's operator $\rho: S \to R$ given by
		\[
		\rho(s)= \dfrac{1}{|G|} \sum_{g \in S} g_S.
		\]
The Reynold's operator is an $R$-linear split surjection, so $S \cong R \oplus \ker \rho$ as $R$-modules. Furthermore, when working with the ring of invariants, one works with symmetric polynomials, which are inherently `nicer' than typical functions. The goal of this section will be to prove the following theorem of Noether:


% \noether28
\begin{restatable}[Noether, 1928]{thm}{noether28} \label{thm:noether28}
The ring $R$ is a noetherian integrally closed domain of Krull dimension $d= \dim_k W$. 
\end{restatable}


\begin{rem}
None of these properties of Noether's Theorem hold for arbitrary subrings of polynomial rings. 
\end{rem}


Recall that the Krull dimension of a $k$-algebra $A$ is the maximal number of algebraically independent elements over $k$. This definition is equivalent to the transcendence degree of the quotient field $Q(A)$ over $k$. If $A$ is the $\C$-algebra given by the coordinate ring of a collection of polynomials, xthen the Krull dimension is the topological dimension of the vanishing set corresponding to $A$; that is, if $A= k[y_1,\ldots,y_m]/(f_1,\ldots,f_r)$, then $\dim A= \dim V(f_1,\ldots,f_r) \subseteq \C^m$. We first show that a polynomial ring is an integral extension of its invariants ring.
 
 
 \begin{lem}
 Let $S=[k_1,\ldots,x_d]$ and $R=S^G$ its ring of invariants. The extension $R \hookrightarrow S$ is integral, i.e. every element of $S$ is a root of a monic polynomial with coefficients in $R$. 
 \end{lem}

\pf For $s \in S$, define $f_s(t):= \prod_{g \in G} (t-gs)$. Then $f_s(t)$ is monic in $t$ and has $s$ as a root since $t-s$ is a factor of $f_s(t)$. The action of $G$ permutes the factors of $f_s(t)$, so it fixes the coefficients of $f_s(t)$. Therefore, $f_s(t) \in R[x]$. \qed \\


This already shows that $\dim R= \dim S= d$ since integral extensions satisfy Going-Up and Going-Down, and thus preserve dimension. We know also that if $A \hookrightarrow B$ is an integral extension of rings and $B$ is finitely generated as an $A$-algebra, then $B$ is finitely generated as an $A$-module (see Atiyah-MacDonald~\cite[Ch.~5]{atiyahmac}). It then follows that


\begin{cor}
Let $S=[k_1,\ldots,x_d]$ and $R=S^G$ its ring of invariants. Then $S$ is a finitely generated $R$-module of rank $|G|$.
\end{cor}

\pf It is enough to show that $S$ is a finitely generated $R$-algebra. Now $S$ is generated over $k$ by $x_1,\ldots,x_d$, so that it is generated over $R$ by $x_1, \ldots, x_d$ as well so that $S$ is a finitely generated $R$-module. It remains to show that $S$ has rank $|G|$. Recall that the rank of a module $M$ over a domain $A$ is $\dim_{Q(A)} M \otimes_A Q(A)$. So we need to compute $\dim_{Q(R)}(S \otimes_R Q(R))$.  But $S \otimes_R Q(R)$ is just $Q(S)$ by integrality. Then any $s \in S$ satisfies an equation $s^p + r_1s^{p-1}+\cdots+r_{p-1}s+r_p=0$, where $r_i \in R$ and $r_p \neq 0$. Then $s(s^{p-1}+r_1s^{p-2}+\cdots+r_{p-1})= -r_p$. Then
	\[
	\dfrac{1}{s}= -\dfrac{1}{r_p} \left( r_1 s^{p-2} + \cdots + r_{p-1} \right).
	\]
If you invert everything in $R$, then you have effectively inverted everything in $s$ by the above. So we need $\dim_{Q(R)} Q(S)$. This a Galois field extension, $Q(S)^G= Q(R)$ so it has degree $|G|$. \qed \\


We now prove that $R$ is noetherian: \\

We know that $S$ is integral over $R$, so that each $x_i$ is integral over $R$. Let $f_1(t), \ldots, f_d(t)$ be monic polynomials with $f_i(x_i)=0$. Let $B \subseteq R$ be the $k$-subalgebra generated by all the coefficients of $f_1,\ldots,f_d$. Then $B$ is finitely generated as a $k$-algebra so that it is noetherian by the Hilbert Basis Theorem. $S$ is integral over $B$, and finitely generated as a $B$-algebra, so finitely generated as a $B$-module. Now $B$ is noetherian and $S$ is a finitely generated $B$-module, every $B$-submodule of $S$ is a finitely generated module. In particular, $R$ must be a finitely generated $B$-module and hence is a noetherian module. Ideals in $R$ are $B$-submodules of $R$ and hence satisfy the ascending chain condition. Hence, $R$ is noetherian. \qed \\



We now prove that $R$ is integrally closed in its quotient field (normal): \\


We know that $Q(S)^= Q(R)$. 


% Field diagram
% 		     Q(S)
%                /        \
% 		S	     \
% 		  \          Q(R)
% 	                R/		


If $x=a/b \in Q(R)$ is integral over $R$, then in particular $x \in Q(S)$ and is integral over $S$. Polynomial rings are integrally closed in their quotient field. So $x \in S$. Hence, $x \in S \cap Q(R)= S \cap Q(S)^G= S^G= R$. When is $R$ a UFD? Always? Never? In between? We know that $S \cong k[x_1,\ldots,x_d]$ is a UFD but $k[u,v]^{C_2}= k[u^2,uv,v^2] \cong k[a,b,c]/(ac-b^2)$ is not a UFD since $u^2 \cdot v^2= (uv)^2$. If we assume that $G$ has no nontrivial linear characters, i.e. no group homomorphisms $G \to k^\times$, then the invariant ring is a UFD. 

\pf Let $r \in R$, then $r \in S$, so we can factor $r= q_1^{a_1} \cdots q_m^{a_n}$, where each $q_i$ is irreducible in $S$ are pairwise non-associate. The $q_i$'s are not necessarily permuted by $G$, but the ideals $(q_1),\ldots,(q_m) \subseteq S$ are. Let $O_1,\ldots,O_e$ be the orbits of the ideals $(q_1),\ldots,(q_m)$ under this action and set
	\[
	Q_j = \prod_{(q_i) \in O_j} q_i^{a_i}
	\]
for $j=1,\ldots,e$. The ideal $(Q_j)$ is stable under the action of $G$, so in particular for any $g \in G$, $gQ_j \in (Q_j)$ so $gQ_j$ is a multiple of $Q_j$. For degree reasons, $gQ_j= \lambda_g Q_j$ for some $\lambda_g \in k^\times$. The function $\lambda: G \to k^\times$ given by $g \mapsto \lambda_g$ (we get one such function for each orbit) and it is a group homomorphism. By assumption, $\lambda$ must be trivial, i.e. $gQ_j= Q_j$ for every $g \in G$ and $j=1,\ldots,e$. Then $Q_j \in R$ for every $j$ and $r= Q_1Q_2 \cdots Q_e$ is a factorization in $R$. Each $Q_j$ is irreducible because it is a product over a single orbit. It is routine, although tedious, to check uniqueness. \qed \\


\begin{ex}
$R= S^G$ is a UFD in the following cases (this is not a complete list, just examples): if $G$ is a nonabelian simple group, e.g. $A_n$ for $n \geq 5$ note that $A_5$ is the $\cI \cong A_5$ group, then we see that the $E_8$ singularity defined by $x^2+y^3+z^5=0$ is a UFD. Or $G= [G,G]$, i.e. every element is a commutator or a product of commutators (in $k^\times$ they are forced to be abelian so go to 1), such groups are called perfect. OR $k= \Q$ and $G$ has odd order (so then does every element) and every image of every element must then be of odd order, which there are none. One can generalize this to $|k|= p^e$ and $\gcd(|G|,|k^\times|=q-1)$. 
\end{ex}


Having proved Noether's Theorem, our next goal is the Hochster-Eagon Theorem that $R= S^G$, i.e. invariant rings are Cohen-Macaulay. We work with graded rings (recall that $k[x_1,\ldots,x_n]= \bigoplus_p k[x_1,\ldots,x_n]_p$, the $p$th graded piece consisting of homogeneous polynomials of degree $p$ and our group actions preserve degree so $k[x_1,\ldots,x_n]^= \bigoplus_p k[x_1,\ldots,x_n]_p^G$. Let $A= \bigoplus_{i=1}^\infty A_i= A_0 \oplus A_1 \oplus \cdots$ be a graded ring with $A_0= k$ a field. Sometimes [Bruns-Herzog] such rings are called *local since they have a unique homogeneous maximal ideal $\fm= A_1 \oplus A_2 \oplus \cdots$. 


\begin{dfn}[Homogenous System of Parameters]
A homogeneous system of parameters for $A$ is a sequence of elements $a_1,\ldots,a_t$ such that $A$ is a finitely generated module over the (graded subring) $k[a_1,\ldots,a_t]$. 
\end{dfn}







%%%%

\subsection{Cohen-Macaulay Rings \& Modules}

\begin{dfn}[Homogeneous]
Let $A= \bigoplus_{i=0}^\infty A_i$ be an $\N$-graded ring with $A_0=k$ a field. An element $a \in A$ is homogeneous if it lies in a unique $A_i$.
\end{dfn}


\begin{dfn}[Homogeneous System of Parameters]
Let $A= \bigoplus_{i=0}^\infty A_i$ be an $\N$-graded ring with $A_0=k$ a field. A sequence $a_1,\ldots,a_m$ of homogeneous elements is called a homogeneous system of parameters (hsop) if $m = \dim A$ and $A$ is a finitely generated module over the graded subring $k[a_1,\ldots,a_n]$. Equivalently, $A/(a_1,\ldots,a_m)$, where $k= \dim A$, is a finite dimensional $k$-vector space. 
\end{dfn}


\begin{ex} \hfill
\begin{enumerate}[(i)]
\item Consider the polynomial ring $A= k[x^2,xy,y^2]$. We claim $\{x^2,y^2\}$ is a hsop. Consider $B= k[x^2,y^2] \subseteq A$. As a $B$-module, $A$ is generated by $1$ and $xy$. Equivalently, $A \cong k[a,b,c]/(ac-b^2)$, $\{a,c\}$ is a hsop as $A/(a,c) \cong k[a,b,c]/(ac-b^2,a,c) \cong k[b]/(b^2)$. 

\item Consider the ring $A= k[x^2,x^3,y,xy]$. Note that $k[x,y]$ is an integral extension of $A$ since $x$ is a root of $t^2-x^2$. Therefore, $\dim A=2$. Now $\{x^2,y \}$ is a hsop: if $B= k[x^2,y]$, then $A= B1 + Bx^3+Bxy$. 
\end{enumerate}
\end{ex}


\begin{thm}[Noether]
Let $k$ be a field and $A$ be a finitely generated $k$-algebra. Then there exist homogeneous elements $a_1,\ldots,a_d$ which are algebraically independent over $k$ and such that $A$ is a finitely generated module over $k[a_1,\ldots,a_d]$ (which is isomorphic to a polynomial ring). 
\end{thm}

It is worth noting that if $k$ is infinite, then there is a probabilistic algorithm for getting the $a_i$'s. In this case, we could take for the $a_i$'s a `sufficiently general' $k$-linear combination of the generators of $A$. This is Noether's Normalization Lemma. Furthermore, the integer $d$ is uniquely determined; it is the Krull dimension of $A$. In the case that $A$ is an integral domain, $d$ is also the transcendence degree of the field of fractions of $A$ over $k$. 


\begin{dfn}[Cohen-Macaulay (CM)]
If $A= \bigoplus_{i=0}^\infty A_i$ is an $\N$-graded ring with $A_0=k$ a field, then if for some (equivalently, any) hsop $a_1,\ldots,a_d$, $A$ is a free module over the Noether normalization $k[x_1,\ldots,x_d]$, we say that $A$ is Cohen-Macaulay (CM). 
\end{dfn}


\begin{ex} \hfill
\begin{enumerate}[(i)]
\item Let $A:= k[x^2,xy,y^2] \supseteq B:= k[x^2,y^2]$. Observe $A= B1+Bxy$ and $\{1, xy\}$ are $B$-linearly independent. Noting also that $1, xy$ are free generators, i.e. $A$ is free over $B$, we see that $A$ is CM.

\item Let $A:= k[x^2,x^3,y,xy] \supseteq B:= k[x^2,y]$. Observing that $A= B1 + Bx^3+Bxy$ but $y(x^3)-x^2(xy)=0$, we see that $A$ is not CM. 

\item Define $A_1= k[x^4,x^3y,x^2y^2,xy^3,y^4]$ and $A_2= k[x^4,x^3y,xy^3,y^4]$. Only one of these is CM, the other is not. (Which?!)
\end{enumerate}
\end{ex}


\begin{dfn}[Maximal Cohen-Macaulay (MCM)]
Let $A$ be as above and $M$ be a finitely generated $A$-module. We say that $M$ is maximal Cohen-Macaulay (MCM) if for some (equivalently, any) hsop, $M$ is a free module over the Noether normalization. 
\end{dfn}


So $A$ is a CM ring if and only if it is a maximal MCM module over itself. 


\begin{rem}
These are not the typical definitions of CM or MCM. They are equivalent, of course. However, one usually begins by considering the local ring case rather than the graded case. However, we choose this approach for clarity and consistency of approach with the rest of the notes.
\end{rem}


\begin{ex} \hfill
\begin{enumerate}[(i)]
\item Let $A= k[x_1,\ldots,x_d]$ be a polynomial ring and $M$ a finitely generated $A$-module. When is $M$ a maximal Cohen-Macaulay module? We know that $\{x_1,\ldots,x_d\}$ form a hsop. But then $M$ is MCM if and only if it is free over $k[x_1,\ldots,x_d]= A$. Therefore for polynomial rings, the MCM rings are the free modules. 

\item Let $A= k[x^2,xy,y^2]$, and let $I= (x^2,xy)A$ be an ideal of $A$. We claim that $I$ is MCM as an $A$-module. Notice that $I$ is isomorphic to the $A$-submodule of $k[x,y]$ generated by $x$ and $y$. Furthermore, $k[x,y] \cong A \oplus (x,y)A$ as $A$-modules. Notice that $\{x^2,y^2\}$ is a hsop for $k[x,y]$. Since polynomial rings are CM, $k[x,y]$ is a free module $k[x^2,y^2]$-module. Then so too must its summands be free. This shows $(x,y)A$ is a free $k[x^2,y^2]$-module as well. 
\end{enumerate}
\end{ex}


As noted, we have taken a non-standard approach to CM and MCM rings. We shall still need alternative definitions for these concepts in places, so we approach them here.


\begin{dfn}[$M$-Regular]
Let $(A,\fm)$ be a local ring, where $\fm$ is the maximal ideal, and $M$ is a finitely generated $A$-module. A sequence $a_1,\ldots,a_t \in \fm$ is an $M$-regular sequence if
	\begin{itemize}
	\item $a_1$ is a nonzerodivisor on $M$
	\item $a_{i+1}$ is not a zero divisor on $M/(a_1,\ldots,a_i)$ for $i=1,\ldots,t-1$
	\end{itemize}
\end{dfn}


\begin{dfn}[Depth]
Let $(A,\fm)$ be a local ring. The depth of a finitely generated $A$-module $M$ is the length of the longest possible $M$-regular sequence.
\end{dfn}


It is known that the depth $\dep M$ is bounded above by $\dim A$, and $M$ is MCM if and only if equality occurs. 


\begin{thm}[Hochster-Eagon] \label{thm:hocheagon}
Let $G$ be a finite subgroup of $\GL(W)$, and assume $|G| \neq 0$ in $k$. Then the invariant ring $k[W]^G$ is a CM ring. 
\end{thm}

\pf Recall Reynold's operator $\rho: k[W]  \twoheadrightarrow k[W]^G$. Let $f_1,\ldots,f_d$ be a hsop in $k[W]^G$. Let $B= k[f_1,\ldots,f_d]$, $R:= k[W]^G$, and $S:= k[W]$.  Then $B \subseteq R \subseteq S$, and $S$ is a finitely generated $R$-module. But then $S$ is a finitely generated $B$-module. Therefore, $f_1,\ldots,f_d$ form a hsop in $S$. Since polynomial rings are CM, $S$ is a free $B$-module. The Reynold's operator makes $R$ into a direct summand of $S$, as $R$-modules. Hence, the Reynold's operator turns $R$ into a $B$-direct summand of a free $B$-module. Therefore, $R$ is free over $B$, as desired. \qed \\


Note that the assumption on $|G|$ is necessary: if we take $C_4$ act by index permutation on $\F_2[x_1,x_2,x_3,x_4]$, then $\F_2[x_1,\ldots,x_4]^{C_4}$ is not CM. This was shown by Bertin in 1967, see Neusel-Smith. Finally, notice that the proof of Theorem~\ref{thm:hocheagon} holds for any $R$-direct summand of $S$. In other words, writing $S= R \oplus M_1 \oplus \cdots \oplus M_r$ as a direct sum of $R$-modules, then each $M_i$ is a MCM $R$-module. What are the $R$-direct summands of $S$? This is a central question of these notes, which we shall spend most of the remaining text trying to answer. 


Recall that if $\chi: G \to k^\times$ is a linear character, then we have the semi-invariants $k[W]^G_\chi:= \{ f \in k[W] \colon gf= \chi(g)f \; \text{ for all } g \in G\}$. We claim that $k[W]^G_\chi$ is an $R$-direct summand of $k[W]$. Define a `fancy' Reynold's operator $\rho_\chi: k[W] \to k[W]^G_\chi$ by 
	\[
	\rho_\chi(f):= \dfrac{1}{|G|} \sum_{g \in G} \chi(g)^{-1} gf.
	\]
Though it is not immediately obvious, we can show that $\im \rho_\chi \subseteq k[W]^G_\chi$ as follows:
	\[
	\begin{split}
	h \rho_\chi(f)&= \vdots \\
	&= \vdots \\
	&= \text{sum over } h^{-1}g \text{see Reynold's proof} \\
	&= \chi(h) \rho_\chi(f).
	\end{split}
	\]
Furthermore, the map $\rho_\chi$ is $R$-linear and splits the inclusion $k[W]^G_\chi \subseteq k[W]$. In other words, $\rho_\chi$ fixes $k[W]^G_\chi$. This shows that each semi-invariant is an $R$-direct summand of $k[W]$. But that is not all! 



\subsection{Isotypical Components}

Isotypical components are the `jazzy' versions of semi-invariants for higher dimensional representations. Let $\chi: G \to k^\times$ be a linear character and $k[W]^G_\chi$ be its ring of semi-invariants. We know that $G$ acts on the polynomial ring $k[W]$ in a way that preserves degrees. But then $G$ acts on each graded piece $k[W]_t$. By Maschke's Theorem, each $k[W]_t$ decomposes into irreducible representations. Fix an irreducible representation $\rho$, and let $k[W]_\rho^G$ be the direct sum of all appearances of $\rho$ in the decomposition above, each in its appropriate degree. This is the isotypical component of $k[W]$ corresponding to $\rho$. `Obviously', 
	\[
	k[W] \cong \bigoplus_{\rho \text{ irred rep}} k[W]_\rho^G. 
	\]
SO each $k[W]^G_\chi$ is an $R$-module, and by the proof of Theorem~\ref{thm:hocheagon}, this is a MCM $R$-module. Though this technically answers our question, it does not held up gain any deeper understanding. 




























