% !TEX root = ../../mat830_mckay.tex
\newpage
\section{Group Representations \& Characters}

Our final goal for this section will be \mc original observation that the character tables of the binary polyhedral groups `are' the extended A-D-E diagrams. We begin with an introduction to group representations.


\begin{dfn}[Representation]
Let $G$ be a group. A (complex) representation of $G$ is a group homomorphism
	\[
	\rho: G \to \GL_n(\C),
	\]
for some $n \geq 1$. We call $n$ the dimension of $\rho$. We call the representation $G \to \C^*=\GL_1(\C)$ given by $g \mapsto 1$ for all $g \in G$ the trivial representation. 
\end{dfn}


We will identify $\GL_n(\C)$ as the automorphism group of $\C^n$, i.e. invertible linear maps. In this way, a representation is equivalent to an action of $G$ on $\C^n$. Write $\rho_g$ for the linear operator $\rho(g): \C^n \to \C^n$. Avoiding a choice of basis, we write $\rho: G \to \GL(V)$ for a vector space $V$. Often, we will not distinguish between $\rho$ and $V$, unless doing so would cause confusion. 


\begin{rem}
Recall the group algebra $\C[G]$ is the $\C$-vector space spanned by the elements of $G$,
	\[
	\C[G]= \left\{ \sum_{g \in G} \alpha_g g \;\bigg|\; \alpha_g \in \C \right\},
	\]
with addition given componentwise and multiplication given by $(\alpha\beta)(gh)$, extended by linearity. Suppose $M$ is a finitely generated $\C[G]$-module, then it is also a finitely generated $\C$-module, i.e. a $\C$-vector space. Therefore, $M \cong \C^n$, as vector spaces. Multiplication by group elements defines linear operators $(M \ma{\cdot g} M) \in \GL(M)\cong\GL_n(\C)$. Therefore, we obtain a map $\rho: G \to \GL_n(\C)$ given by $g \mapsto (M \ma{g} M)$, i.e. a representation. Conversely, a representation $V$ is equivalent to a $\C[G]$-module. Therefore, the following are equivalent
	\begin{itemize}
	\item a representation of a group $G$
	\item a $\C[G]$-module
	\item an action of $G$ on $\C^n$
	\end{itemize}
\end{rem}


The direct sum of two representations $\rho: G \to \GL_n(\C)$, $\rho': G \to \GL_m(\C)$ is $\rho \oplus \rho' : G \to \GL_{n+m}(\C)$ given by 
	\[
	g \mapsto \left(\begin{tabular}{c|c} $\rho(g)$ & 0 \\ \hline 0 & $\rho'(g)$ \end{tabular}\right).
	\]

% action of $G$ on V oplus V' which stabilizes the subspaces V,V'

\begin{dfn}[Indecomposable]
If $\rho$ cannot be written as a direct sum of two representations, then we call the representation indecomposable. Otherwise, we call the representation decomposable. 
\end{dfn}


If $\rho$ is decomposable, there are invariant, i.e. stabilized, subspaces of the vector space $V \oplus V'$. 


\begin{dfn}[Irreducible]
We say that a representation $\rho: G \to \GL(V)$ is irreducible if $\rho$ has no invariant subspaces, i.e. no submodules other than $\{0\}$ and $V$. Otherwise, we say that $\rho$ is reducible.
\end{dfn}


Clearly, a decomposable representation must be reducible, which immediately gives the following by contrapositive. 

\begin{thm}
Any irreducible representation is indecomposable. 
\end{thm}


\begin{ex}
Let $G=S_3$. What are the representations of $S_3$? There is always the trivial representation $1: S_3 \to \C^*$ given by $\sigma \mapsto 1$ for all $\sigma \in S_3$. We also have the sign (or alternating) representation $a: S_3 \to \C^\times$ given by $\sigma \mapsto (-1)^{\text{sign }\sigma}$, which restricts to an injection from $A_3$ to 1, i.e. $S_3 \setminus A_3$ injects to $-1$ under $a$. We know that every permutation can be represented by a matrix given by mapping a permutation $\sigma$ to the result of the permutation $\sigma$ acting on the rows of $I_3$. For example,
	\[
	(2\;3) \mapsto \three{1}{0}{0}{0}{1}{0}{0}{0}{1}.
	\]
This gives a homomorphism $S_3 \to \GL_3(\C)$, called the natural representation. This defines an action of $S_3$ on $\C^3$ by permutation of basis, i.e. $\sigma(\{z_1,z_2,z_3\})= \{z_{\sigma^{-1}(1)}, z_{\sigma^{-1}(2)}, z_{\sigma^{-1}(3)}\}$. 

Clearly, the trivial representation is both indecomposable and irreducible. The sign representation has dimension one, so it is both indecomposable and irreducible. The natural representation has stable subspaces, namely the one spanned by $(1,1,1)$, so that it cannot be indecomposable, i.e. the natural representation is decomposable. But then the natural representation is also reducible. We can also create submodule/subrepresentations by `modding out.' For example, define $V=\{ (z_1,z_2,z_3) \in \C^3 \colon z_1+z_2+z_3=0\}$---the natural representation modulo the trivial representation, with the permutation action. This is called the standard representation. This space has dimension two and one can check the permutation representation is isomorphic to $1 \oplus V$. \xqed
\end{ex}


\begin{thm}[Maschke's Theorem]
Every indecomposable representation over $\C$ of a finite group is irreducible. Therefore, a representation over $\C$ of a finite group is indecomposable if and only if it is irreducible.
\end{thm}

\pf Suppose that $V$ is a representation of $G$ and $W \subseteq V$ is a subrepresentation, i.e. a $G$-stable subspace. Fix a linear projection $\pi: V \twoheadrightarrow W$, and $G$-linearize it:
	\[
	\widetilde{\pi}(v) = \dfrac{1}{|G|} \sum_{g \in G} (g\pi g^{-1})(v).
	\]
Now notice we have
	\[
	\begin{split}
	h\widetilde{\pi}(v)&= \dfrac{1}{|G|}h \sum_g (g\pi g^{-1})(v) \\
	&= \dfrac{1}{|G|} \sum_g hg\pi g^{-1}h^{-1}hv \\
	&= \dfrac{1}{|G|} \sum_{hg} (hg)\pi(hg^{-1})(hv) \\
	&= \widetilde{\pi}(hv).
	\end{split}
	\]
Therefore, $h\widetilde{\pi}(v)=\widetilde{\pi}(hv)$ so $\widetilde{\pi}$ is $G$-linear. It is routine to verify that $\widetilde{\pi}$ fixes $W$, and we know $\widetilde{\pi}$ projects $V$ onto $W$. Hence, $V \cong W \oplus \ker \widetilde{\pi}$. But then $V$ is reducible. \qed \\


\begin{rem}
This works over any field with $|G| \neq 0$. Another way to say this is that the group algebra $\C[G]$ is semisimple, i.e. short exact sequence of $\C[G]$-modules splits. 
\end{rem}



\subsection{Characters}

\begin{dfn}[Character]
Let $\rho?L G \to \GL_n(\C)$ be a representation of $G$. The character of $\rho$ is $\chi_\rho:= \text{tr}\circ \rho$, i.e. the composition
%	\[
%	\begin{tikzcd}
%	G \arrow{r}{\rho} \arrow[bend=50ex]{rr}{\chi_p} & \GL_n(\C)  \arrow{r}{\text{tr}} & \C 
%	\end{tikzcd}
%	\]
When the representation is apparent, we denote this simply as $\chi$.
\end{dfn}


Observe that $\chi_\rho$ is \emph{not} generally a homomorphism as the trace is not generally multiplicative, i.e. $\text{tr}(AB) \neq \text{tr}(A)\text{tr}(B)$. If $n=1$, then clearly $\chi_\rho$ is a homomorphism. Now while the trace map is not generally multiplicative, we do have that $\text{tr}(AB)=\text{tr}(BA)$. More generally, $\text{tr}(\cdot)$ is invariant under cyclic permutation of products. Therefore, $\chi_\rho$ is a \emph{class function}, i.e. $\chi_\rho$ is constant on conjugacy classes:
	\[
	\chi_\rho(g^{-1}hg)= \text{tr}(\rho(g^{-1}hg))=\text{tr}(\rho(g)^{-1}\rho(h)\rho(g))= \text{tr}(\rho(h))=\chi_\rho(h).
	\]


\begin{ex} %cf exercise
Take $G=S_3=\{(1),(1\;2),(2\;3),(1\;3),(1\;2\;3),(1\;3\;2)\}$. The conjugacy classes of $S_n$ are classified by cycle type---corresponding to integer partitions of $n$. These are
	\[
	\{(1)\}, \{(1\;2),(2\;3),(1\;3)\}, \{(1\;2\;3),(1\;3\;2)\}.
	\]
We shall create a table for the characters of $S_n$. We only need one column per conjugacy class, and one row per representation. The trivial representation takes every $\sigma$ to the identity, so $\chi_{\text{triv}}(\sigma)=1$ for all $\sigma$. We know for the alternating representation that 
	\[
	a(\sigma)= 
	\begin{cases}
	1, & \sigma \text{ even} \\
	-1, & \sigma \text{ odd}
	\end{cases}
	\]
Therefore, $\chi_{\text{a}}$ is the same as $\chi_{\text{triv}}(\sigma)=1$. For the permutation representation, we have
	\[
	1 \mapsto \three{1}{}{}{}{1}{}{}{}{1} \quad (1\;2) \mapsto \three{}{1}{}{1}{}{}{}{}{1} \quad (1\;2\;3) \mapsto \three{}{}{1}{1}{}{}{}{1}{}
	\]
We knew also that the permutation representation was isomorphic to the standard representation summed with the trivial representation. Since the trace of a block matrix is the sum of the traces, we know that $\chi_{\text{perm}}= \chi_{\text{std}}+\chi_{\text{triv}}$. We can then subtract to find $\chi_{\text{std}}$, making its row on the table below redundant. 
	\begin{table}[h]
	\centering
	\begin{tabular}{c|rrr}
	& $(1)$ & $(1\;2)$ & $(1\;2\;3)$ \\ \hline
	$\chi_{\text{triv}}$ & 1 & 1 & 1 \\
	$\chi_{\text{a}}$ & 1 & $-1$ & 1 \\
	$\chi_{\text{perm}}$ & 3 & 1 & 0 \\
	$\chi_{\text{std}}$ & 2 & 0 & $-1$
	\end{tabular}
	\end{table}
If we remove the redundant $\chi_{\text{perm}}$ row, we obtain the following table:
	\begin{table}[h]
	\centering
	\begin{tabular}{c|rrr}
	& $(1)$ & $(1\;2)$ & $(1\;2\;3)$ \\ \hline
	$\chi_{\text{triv}}$ & 1 & 1 & 1 \\
	$\chi_{\text{a}}$ & 1 & $-1$ & 1 \\
	$\chi_{\text{perm}}$ & 3 & 1 & 0 \\
	$\chi_{\text{std}}$ & 2 & 0 & $-1$
	\end{tabular}
	\end{table}
We make the following observations:
	\begin{enumerate}[1.]
	\item The table is square, and the number of characters is the number of conjugacy classes.
	\item The columns are orthogonal.
	\item The rows are orthogonal if one weights each column by the number of elements in that class, e.g.
		\[
		\langle \chi_{\text{triv}}, \chi_{\text{std}} \rangle = 1(1 \cdot 2 ) + 3(1 \cdot 0) + 2(1\cdot -1)=0.
		\]
	\item The first column yields the dimension of $\rho$. In general, $\chi_p(1)=\text{tr}(\rho(1))= \text{tr}(I_n)=n$.
	\item The sum of the squares of the 1\tss{st} column is $6=|S_3|$. 
	\end{enumerate}


\end{ex}






\begin{prop}
Let $G$ be a finite group, $\rho$ a finite dimensional representation of $G$ and $\chi$ its corresponding character. Then
	\begin{enumerate}[(i)]
	\item $\chi$ is a class function.
	\item $\chi(1)=n$.
	\item The characters of a direct sum of representations is the sum of the characters.
	\item The character of a tensor product of representations is the product of the characters.	
	\item $\chi(g^{-1})=\ov{\chi(g)}$
	\item If $|g|=k$, then the eigenvalues of the matrix $\rho_g$ are powers of the $k$\tss{th} roots of unity, and $\chi(g)$ is a sum of such things.
	\end{enumerate}
\end{prop}


%4
Recall that if $V$ and $W$ are vector spaces with basis $\{e_1,\ldots,e_n\}$ and $\{f_1,\ldots,f_m\}$, respectively, then $V \otimes W$ is the vector space with basis $\{e_i \otimes f_j \}_{i=1,\ldots,n; j=1,\ldots,m}$ and scalar multiplication $\alpha(e_i \otimes f_j) = \alpha e_i \otimes f_j= e_i \otimes f_j$. If $V$ and $W$ carry actions of $G$, then so does $V \otimes W$ by $g(v\otimes w)= g(v) \otimes g(w)$. 
%5 exercise
%6 
If $g^k=1$, then $(\rho_g)^k=I_n$ so the minimal polynomial of $\rho_g$ divides $x^k-1$. Therefore, its roots are roots of unity. The trace of a matrix is the sum of its eigenvalues.




\subsection{Orthogonality Relations}


Let $\cH$ denote the set of all class functions $G \to \C$. This contains the characters of $G$. Define a Hermitian inner product on $\cH$
	\[
	\langle \phi, \psi \rangle:= \dfrac{1}{|G} \sum_{g \in G} \ov{\phi(g)} \psi(g).
	\]
%Compare this with the weighted dot product for $S_3$.



\begin{thm}
The irreducible characters, i..e the characters of irreducible representations, are an orthonormal basis for $\cH$ with respect to this inner product. In particular,
	\[
	\langle \chi_\rho, \chi_{\rho'} \rangle =
	\begin{cases}
	1, & \rho \cong \rho' \\
	0, & \rho \neq \rho'
	\end{cases}
	\]
\end{thm}

\pf The proof will proceed in eight steps.
	\begin{enumerate}[(i)]
	\item For any representation $V$, the fixed subspace $V^G:=\{v \in V \colon gv=v \text{ for all }g \in G\}$ is a subrepresentation of $V$. There is a natural projection
		\[
		\begin{tikzcd}
		\pi: V \arrow{r} & V^G \subseteq V \\
		v \arrow[mapsto]{r} & \displaystyle\dfrac{1}{|G|} \sum_{g \in G} gv.
		\end{tikzcd}
		\]
	\item Compute the trace of $\pi$. First, extend a basis for $V^G$ to a basis for $V$. Then
%		\[
%		[\pi] = \begin{pmatrix} 1 diagon then 0 then 0s
%		\]
		and so $\text{tr}(\pi)= \dim V^G$. Now the trace of a sum is the sum of the traces, so 
			\[
			\text{tr}(\pi)= \text{tr}\left( \dfrac{1}{|G|} \sum_{g \in G} g(\cdot) \right) = \dfrac{1}{|G|} \sum_{g \in G} \text{tr}(\rho_g)= \dfrac{1}{|G|} \sum_{g \in G} \chi_\rho(g).
			\]
		In other words, $\dim V^G$ is the average value of $\chi_\rho$. 
	
	\item For representations $V$ and $W$, we have
		\[
		\begin{tikzcd}
		\Hom_\C(V,W)=\{\text{linear maps }V \to W\} \\ %supset
		\Hom_G(V,W) = \left\{ G\text{-linear maps, i.e. }gf(v)=f(gv)\right\}
		\end{tikzcd}
		\] 
	Now $\dim \Hom_\C(V,W)= \dim V \cdot \dim W$. However, what is $\dim \Hom_G(V,W)$? 
	
	
	\item We know that $\Hom_\C(V,W)$ is again a representation of $G$: for $g \in G$, $f: V \to W$ a linear map, define $(gf)(v):= g(f(g^{-1}v))$. %has to satisfy (hg)(f)= h(g(f)) and taking h=g^{-1}, the formula follows.
	
	\item Now $\Hom_\C(V,W)^G= \{ f \in \Hom_\C(V,W) \colon gf=f \text{ for all }g \in G\}$. That is, $\{ f \colon (gf)(v)=f(v) \text{ for all }g \in G, v \in V\}$. which is $\{f \colon g(f(g^{-1}(v)))=f(v) \text{ for all }g,v\}$, rearranging is $f(g^{-1}(v))= g^{-1}(f(v))$ for all $g$, which is $\Hom_G(V,W)$. 
	\end{enumerate}
So if $V$ and $W$ are irreducible, 
	\[
	\dim \Hom_G(V,W)= 
	\begin{cases}
	1, & V \cong W \\
	0, & V \not\cong W
	\end{cases}
	\]
and on the other hand, $\dim \Hom_G(V,W)= \dim(\Hom_\C(V,W)^G)$ by 5, which is $=\dim((V^* \otimes_\C W)^G)$, which is the average value of $\chi_{V^* \otimes W}$ which is the average value of $\ov{\chi_V} \chi_W$ which is $\langle \chi_V, \chi_W \rangle$. 








\newpage





% PP or list?


Consequently, the characters determine the representations; that is, $\rho \cong \rho'$ if and only if $\chi_\rho = \chi_{\rho'}$. 


The number of irreducible representations of $G$ is equal to the number of conjugacy classes. [Because $\cH$ has a basis given by the characteristic functions of the conjugacy classes.]


A representation $\rho$ is irreducible if and only if $\langle \chi_\rho,\chi_\rho\rangle=1$. [For any $\rho$, Maschke's Theorem allows one to write $\rho \cong \rho_1^{a_1} \oplus \cdots \oplus \rho_r^{a_r}$, where the $\rho_i$ are distinct, irreducible, and $a_i$ is its multiplicity. But then $\chi_\rho= a_1\chi_{\rho_1} + \cdots + a_r \chi_{\rho_r}$. But then 
	\[
	\langle \chi_\rho, \chi_\rho \rangle = \sum_{i,j} a_i a_j \langle \chi_{\rho_i}, \chi_{\rho_j} \rangle= \sum_{i=1}^r a_i^2.
	\]
Therefore, $\rho=\rho_i$ must be irreducible. The other direction follows straight from the theorem.


The multiplicity of an irreducible representation $\rho_i$ in a given representation is $\langle \chi_{\rho_i}, \chi_\rho \rangle$. % Using the same trick.


% End list



% 09/11/2018

\begin{dfn}[Regular Representation]
The regular representation of $G$ is a $\C$-vector space $\C[G]$. Equivalently, $\C[G]$, as a module over itself, or $G \to \GL(\C[G])$.
\end{dfn}


Recall $\C[G]$ has a basis $\{g \in G\}$. The action of $h \in G$ is given by $g \mapsto hg$, i.e. $h$ permutes basis elements. 


\begin{prop}
Every irreducible representation $V$ appears as a direct summand in the regular representation, with multiplicity equal to its dimension, i.e.
	\[
	\C[G] \cong \bigoplus_{i=1}^r V_i^{\dim V_i},
	\]
where $V_1,\ldots, V_r$ are the irreducible representations. In particular, 
	\[
	|G|= \sum_{i=1}^r (\dim V_i)^2= \sum_{i=1}^r (\chi_i(1))^2.
	\]
\end{prop}

\pf L.T.R. \qed \\






\begin{cor}
$G$ is abelian if and only if every representation is 1-dimensional. 
\end{cor}


\pf $G$ is abelian if and only if every conjugacy class is a singleton if and only if there are $|G|$ classes if and only if there are $|G|$ irreducibles if and only if all the representations have dimension 1. \qed \\









\begin{ex}
\begin{enumerate}[(i)]
\item $G=S_3$. We found three irreducible representations: $\chi_{\text{triv}}, \chi_{\text{akt}}, \chi_{\text{std}}$. Since $1^2+1^2+2^2=6=|S_3|$, this must be all the irreducible representations for $S_3$. 

\item Let $G= C_n= \langle x \colon x^n=1\rangle$. Every irreducible is a map $G \to \C^\times$ completely determined by the image of $x$. Since the map is a morphism, $1 \mapsto 1$, which implies that the image of $x$ must be an $n$\tss{th} root of unity. But then we obtain
	\[
	\begin{split}
	\rho_k: \; x &\mapsto \omega_n^j \\
	x^r &\mapsto \omega_n^{j_r}
	\end{split}
	\]
for $j=0,\ldots,n-1$, where $\rho_0$ is the trivial representation. Then we have character table
	\[
	\begin{tabular}{cccccc}
	& $\{1\}$ & $\{x\}$ & $\{x^2\}$ & $\cdots$ & $\{x^{n-1}\}$ \\
	$\rho_0$ & 1 &1 & $\cdots$ & 1 \\
	$\rho_1$ & $\omega$ & $\omega^2$ & $\cdots$ & $\omega^{n-1}$ \\
	$\rho_2$ & 1 & $\omega^2$ & $\omega^4$ & $\cdots$ & $\omega^{2(n-1)}$ \\
	$\vdots$ & $\vdots$ & $\vdots$ & $\vdots$ & $\ddots$ & $\vdots$ \\
	$\rho_{n-1}$ & 1 & $\omega^{n-1}$ & $\omega^{2(n-1)}$ & $\cdots$ & $\omega^{(n-1)^2}$
	\end{tabular}
	\]

\item Let $G= S_4$. We have cycle types $1,(1\;2),(1\;2\;3),(1\;2\;3\;4),(1\;2)(3\;4)$, with multiplicity, $1,6,2\binom{4}{3}=8$, $3!=6$, 3, respectively. 
	\[
	\begin{tabular}{cccccc}
	$\chi_{\text{triv}}$ & 1 & 1 & 1 & 1 &1 \\
	$\chi_{\text{alt}}$ & 1 & $-1$ & 1 & $-1$ & 1 \\
	$\chi_{\text{std}}$ & 3 & 1 & 0 & $-1$ & $-1$ \\
	$\chi_{\text{perm}}$ & 4 & 2 & 1 & 0 & 0 \\
	$\chi_{\text{std}} \otimes \chi_{\text{alt}}$ & 3 & $-1$ & 0 & 1 & $-1$ \\
	$R$ & 2 & 0 & $-1$ & 0 & 2
	\end{tabular}
	\]
Is the standard representation irreducible> We have $\langle \chi_{\text{std}}, \chi_{\text{std}} \rangle = \dfrac{1}{|G|} (1\cdot 3^2+6 \cdot 1^2 + 6 \cdot 0^2+6(-1)^2+3(-1)^3)= \dfrac{1}{24} \cdot 24=1$, so yes. Are we done? Well, we have $1^2+1^2+3^2=11<24$, so no. A sneaky trick is to tensor with the known representations. Tensoring with the trivial one does nothing so we proceed with the others. The tensor of the standard with the alternating representation is 3-dimensional and irreducible by the same calculation. So now $1^2+1^2+3^2+3^2=20$, missing 4. So missing one two dimensional or 4 1-dimensional. In either case, there is (at least one) 2-dimensional representation, say $R$. So the first row of the entry for $R$ must be 2. Call the other entries $a$, $b$, $c$, and $d$ respectively. Using the orthogonality relations, one finds a system of four equations and four unknowns only to find $a=c=0$, $b=-1$, $d=2$. Finally, $\langle \chi_R, \chi_R \rangle= \cdots = 1$, so $R$ is irreducible, as expected. This is then the complete character table. 

\item $G=A_4 \subseteq S_4$. The Class Equation tells that the number of things in each conjugacy class must divide the order of the group. So unlike the situation in $S_4$, $(1\;2\;3)$ is not a conjugacy class (size 8, $|A_4|=12$). ?????????

So $(1\;2\;3)\not\sim (1\;3\;2)$.

So it must split into at least to conjugacy classes. These are 1, $(1\;2\;3)$, $(1\;3\;2)$, $(1\;2)(3\;4)$, of sizes 1, 4, 4, and 3, respectively.
	\[
	\begin{tabular}{ccccc}
	& 1 & $(1\;2\;3)$ & $(1\;3\;2)$ & $(1\;2)(3\;4)$ \\
	$\chi_{\text{triv}}$ & 1 & 1 & 1 & 1 \\
	$\chi_{\text{std}}$ & 3 & 0 & 0 & $-1$ \\
	$R$ & 2 & $-1$ & $-1$ & 2 \\ % call R \chi_M
	\end{tabular}
	\]
We always have the trivial representation. We can always restrict any representation of $S_4$ to $A_4$. Doing so with the alternating representation gives the trivial representation. The standard tensor alternating restricted to $A_4$ is the standard representation on $A_4$. Note one can restrict an irreducible representation and no longer be irreducible. We have $\langle \chi_{\text{std}},\chi_{\text{std}}\rangle =1$, $\langle R,R \rangle= \dfrac{1}{12}(1\cdot 2^2+4(-1)^2+4(-1)^2+3\cdot 2^2)=2$, not irreducible. So the restriction of $R$ to $A_4$ splits into two 1-dimensional representations. So we must split the $R$ row into two, say $U$ and $U'$.
	\[
	\begin{tabular}{ccccc}
	& 1 & $(1\;2\;3)$ & $(1\;3\;2)$ & $(1\;2)(3\;4)$ \\
	$\chi_{\text{triv}}$ & 1 & 1 & 1 & 1 \\
	$\chi_{\text{std}}$ & 3 & 0 & 0 & $-1$ \\
	$U$ & 1 & a & b & c \\
	$U'$ & 1 & $-1-a$ & $-1-b$ & $2-c$
	\end{tabular}
	\]
Once we have $a,b,c$, we know that the rows of $U,U'$ add to the rows of $R$, hence the last row must be what is given above. Linear Algebra gives $a=\omega_3$, $b=\omega_3^2$, and $c=1$. Could we have seen that without Linear Algebra? If we have a quotient of $A_4$, where characters we know, we can restrict along the quotient map. Normal subgroup in $A_4$: $\{1,(1\;2)(3\;4), (1\;3)(2\;4), (1\;4)(2\;3)\}$. The quotient is, having order 3, $C_3$, which has two nontrivial irreducible representations. Let's say $C_3=\langle (1\;2\;3)\rangle$. Then these representations are $(1\;2\;3) \mapsto \omega_3$, $(1\;3\;2) \mapsto \omega_3^2$ and $(1\;2\;3) \mapsto \omega_3^2$, $(1\;3\;2) \mapsto \omega_3$. 
	\[
	\begin{tabular}{ccccc}
	& 1 & $(1\;2\;3)$ & $(1\;3\;2)$ & $(1\;2)(3\;4)$ \\
	$\chi_{\text{triv}}$ & 1 & 1 & 1 & 1 \\
	$\chi_{\text{std}}$ & 3 & 0 & 0 & $-1$ \\
	$U$ & 1 & $\omega$ & $\omega^2$ & 1 \\
	$U'$ & 1 & $\omega^2$ & $\omega$ & 1
	\end{tabular}
	\]

\item Take $G=B\cT \subseteq \SL_2(\C)$, which has a 2-to-1 map $B\cT \to \cT$, where $\cT$ is the tetrahedral group of order 12. We know that $\cT \cong A_4$. Then we know that we can restrict the 4 irreducibles of $A_4$ to $B\cT$. Then the preimage of a conjugacy class in $\cT$ is either a single conjugacy class in $B\cT$ of twice the size, or 2 classes, each of the same size as the original. So to start, just lump the classes together. The classes are 1, $(1\;2\;3)$, $(1\;3\;2)$, $(1\;2)(3\;4)$, of sizes 2, 8, 8, 6.
	\[
	\begin{tabular}{ccccc}
	$\chi_{\text{triv}}$ & 1 & 1 & 1 & 1 \\
	$\chi_{\text{std}}$ & 3 & 0 & 0 & $-1$ \\
	$U$ & 1 & $\omega$ & $\omega^2$ & 1 \\
	$U'$ & 1 & $\omega^2$ & $\omega$ & 1 \\
	$\rho \otimes U$ &
	$\rho \otimes U'$ & 
	\end{tabular}
	\] %left column restriction of
There is also the ``given rep'' $B\cT \hookrightarrow \GL_2(\C)$, which is two-dimensional. Also, $\rho \otimes U$, $\rho \otimes U'$, there are two more 2-dimensional. So we'll have different values in the $(1\;2\;3)$ column, $a$, $\omega a$, $\omega^2 a$ for some $a$. So they are pairwise nonisomorphic. Fact, $\rho$ is irreducible (we had explicit matrix generators for $B\cT$). So $\rho \otimes U$, $\rho \otimes U'$ are too. Then $1^2+3^2+1^2+1^2+2^2+2^2+2^2=24$, and that is all. There are then seven conjugacy classes in $B\cT$. The preimage of $\{1\}$ is $\{\pm 1\}$, the identity matrix. So that class splits in two. Fact: the class $\{(1\;2)(3\;4), (1\;3)(2\;4), (1\;4)(2\;3)\}$ in $\cT$ lifts to a single class of size 6. Then the other two split into two. 
% top row class size first 4s the (123) second 4s (132)
%	\[
%	\begin{tabular}{cccccccc}
%	& 1 & 1 & 4 & 4 & 4 & 4 & 6 \\
%	$\chi_{\text{triv}}$  & 1 & 1 & 1 & 1 & 1 & 1 & 1 \\
%	$\chi_{\text{std}}  & 3 & 3 & 0 & 0 & 0 & 0 & $-1$ \\
%	$U$ & 1 & 1 & $\omega$ & $\omega$ & $\omega^2$ & $\omega^2$ & 1 \\
%	$U'$ & 1 & 1 & $\omega^2$ & $\omega^2$ & $\omega$ & $\omega$ & 1 \\
%	$\rho$ 2 & a & b & c & d & e & f \\
%	$\rho \otimes U$ & a
%	$\rho \otimes U'$ & a
%	\end{tabular}
%	\]
	\[
	\begin{tabular}{cccccccc}
	& 1 & 1 & 4 & 4 & 4 & 4 & 6 \\
	$\chi_{\text{triv}}$  & 1 & 1 & 1 & 1 & 1 & 1 & 1 \\
	$\chi_{\text{std}}$  & 3 & 3 & 0 & 0 & 0 & 0 & $-1$ \\
	$U$ & 1 & 1 & $\omega$ & $\omega$ & $\omega^2$ & $\omega^2$ & 1 \\
	$U'$ & 1 & 1 & $\omega^2$ & $\omega^2$ & $\omega$ & $\omega$ & 1 \\
	$\rho$ & 2 & $-2$ & 1 & $-1$ & 1 & $-1$ & 0 \\
	$\rho \otimes U$ & 2& $-2$ & $\omega$ & $-\omega$ & $\omega^2$ & $-\omega^2$ & 0 \\
	$\rho \otimes U'$ & 2 & $-2$ & $\omega^2$ & $-\omega^2$ & $\omega$ & $-\omega$ & 0
	\end{tabular}
	\]
\end{enumerate}
\end{ex}



Note that the character table does not determine the group. For example $D_4$ and $Q_8$ have the same character table. [Lose a lot passing to conugacy classes.] However, the character table carries a lot of information about the group. For example, if $\rho_1,\ldots,\rho_r$ are the irreducible representations, then for every $i,j$, 
	\[
	\rho_i \otimes \rho_j \cong \bigoplus_{k=1}^r \rho_k^{c_{i,j}^k}
	\]
for some \emph{structure constants} of the group, $c_{i,j}^k$. When $G$ is given to us as a subgroup of $\GL$, it's already interesting to look at
	\[
	\rho \otimes \rho_j = \bigoplus_{i=1}^r \rho_i^{c_{i,j}},
	\]
where $\rho$ is the given representation. Then
	\[
	\chi \chi = \sum c_{i,j} \chi_i
	\]
and we can read the $c_{i,j}$'s from the character table. Back to $B\cT$. We are given $\rho$, the 5th row of the table. Let's decompose $\rho \otimes \text{std}$. We have $\rho \otimes \text{std}: 6,-6,0,0,0,0,0$. Checking carefully and using the properties of $\omega$, this is the sum of $\rho$, $|\rho \otimes U$, and $\rho \otimes U'$. 



\begin{dfn}[\mc Quiver]
Let $G$ be a finite subgroup of $\GL_n(\C)$. The \mc quiver of $G$ has vertices $\rho_1,\ldots,\rho_r$, the irreducible representations of $G$, arrows $m_{ij}$ for $\rho_i \to \rho_j$ if $\rho_i$ appears with multiplicity $m_{ij}$ in $\rho \otimes \rho_j$. 
\end{dfn}


% Insert table from page 19.


Recall $m_{ij}= \dim \Hom_G(V_i, V \otimes V_j)= \langle \chi_i, \chi\chi_j \rangle$ (abstract stuff on boses). We have $G=C_2$ embedded in $\GL_3(\C)$ as $\left\langle \three{-1}{}{}{}{-1}{}{}{}{-1} \right\rangle$. We know the irreducible representations of $C_2= \langle \sigma \colon \sigma^2=1\rangle$. The two representations must be $\sigma \mapsto 1$, $ \sigma \mapsto -1$, call the first $\rho_1$ and the second $\rho_{-1}$. Notice $\rho(\text{given}) \cong \rho_{-1}^{(3)}$ and 
	\[
	\begin{tabular}{c|cc}
	& $\rho_1$ & $\rho_{-1}$ \\ \hline
	$\rho_1$ & $\rho_1$ & $\rho_{-1}$ \\
	$\rho_{-1}$ & $\rho_{-1}$ & $\rho_1$ 
	\end{tabular}
	\]
So $\rho \otimes \rho_1= \rho = \rho_{-1}^{(3)}$, $\rho \otimes \rho_{-1} = \rho_{-1}^{(3)} \otimes \rho_{-1} = \rho_1^{(3)}$. Then we have

% rho_1 on left rho _{-1} left, three curved arrows between them.

Now consider $C_n= \left\langle \two{\omega_n}{}{}{\omega_n^{-1}} \right\rangle \subseteq \SL_2(\C)$. We know the irreducible representations of $C_n: \rho_0,\rho_1,\ldots,\rho_{n-1}$, where $\rho_j$ takes the generator of $C_n$ to $\omega_n^j$. Our given representation is $\rho \cong \rho_1 \otimes \rho_{n-1}$. What is $\rho_j \otimes \rho_k$? It's $\rho_{j+k}$, with $jk$ taken mod $n$. So $\rho \otimes \rho_j= (\rho_1 \oplus \rho_{n-1}) \otimes \rho_j= \rho_{j+1} \oplus \rho_{j-1}$, where again indices are taken mod $n$. 


% \rho_{j-1} right arrow to \rho_j then on right rho_{j+1} with left arrow to \rho_j. 

We get the above diagram for every $j$. 

% Give circular diagram rho_0 at top of clock and go around to \rho_{n-1} with arrows back and forth between each node. 



Example

	\[
	\begin{tabular}{l |rrrrr}
	$D_4$ & $\{1\}$ & $\{x^2\}$ & $\{x,x^3\}$ & $\{y,x^2y\}$ & $\{y,x^3y\}$ \\ \hline
	$\beta_{++}$ & 1 & 1 & 1 & 1 & 1 \\
	$\beta_{+-}$ & 1 & 1 & 1 & $-1$ & $-1$ \\
	$\beta_{-+}$ & 1 & 1 & $-1$ & $-1$ & $-1$ \\
	$\beta_{--}$ & 1 & 1 & $-1$ & $-1$ & 1 \\
	$\rho$ & 2 & $-2$ & 0 & 0 & 0 
	\end{tabular}
	\]


where $\rho_{\pm\pm}(x)=\pm1$, $\beta_{\pm\pm}(y)=\pm1$, and $\rho$ is the ``geometric'' representation as symmetries of an $n$-gon in $\C^2$.

Now let's compute the \mc quiver of $D_4$ with respect to $\rho: D_4 \hookrightarrow \GL_2(\C)$. 

% \rho on left \otimes \beta_{--} with
	
	\[
	\begin{split}
	\rho \otimes \beta_{++} &\cong \rho \\
	\rho \otimes \beta_{+-} &\cong \rho \\
	\rho \otimes \beta_{-+} &\cong \rho \\
	\rho \otimes \beta_{--} &\cong \rho \\
	\rho \otimes \rho &\cong \beta_{++} \oplus \beta_{+-} \oplus \beta_{-+} \oplus \beta_{--}
	\end{split}
	\]

% 4 diagram clockwise with bpp at top, rho center, with arrows going each way.



Then picture

% BT \subseteq \SL_2(\C) below it 
% Below it U updown arrow \rho \otimes U \updown arrow std, then row beneath with left/righa arrows in order triv rho std rho, rho \otimes U' then U'

% Give same diagram beneath with dots alone except for trivial, label BO.

% Give another with left right arrows, node above 3rd dot, trivial at end, row has 8 entries, including trivial. 


\mc observed that the arrows in the \mc quiver of the binary tetrahedral groups come in opposing pairs, no more than one between any two vertices, and if you remove the trivial representation, one obtains an ADE Dynkin diagram. 


The first two parts are relatively simple to prove without knowing the classification. For example, $m_{ji}= \langle \chi_j,\chi\chi_i \rangle$. Now $\chi$ is the character of $G \hookrightarrow \SL_2(\C)$ is self-adjoint since it is in $\SL_2$. But then $m_{ji}= \langle \chi_j,\chi\chi_i \rangle= \langle \chi_j\chi,\chi_i \rangle= \langle \chi_i, \chi,\chi_j \rangle=m_{ij}$. 


Our next goal is to give a uniform proof (meaning without classification) of \mc's observation about ADE diagrams. 




\subsection{ADE and Extended ADE Diagrams}


A list:

% left (A_n) horizontal dots, one above, lines connecting for triangle. on far right (\tilde{A}_n) circle to vertex

% Dn same as D_n with one extra off far right to make 'symmetric' on far right (\tilde{D}_n)

% (E_6) . - . - (two above this, top circled) - . - .

% E7 7 dots as above, far left circled, one extra above 4th one

% E_8 8 horizonal, last circled, extra above third.

% Note that n \geq 0 and for consistentcy \tilde{A}_0 single self loop dot.


The extended ADE diagrams have one extra vertex, circled. Then have $n+1$ vertices. They have the weird property that \dots

\begin{lem} \label{lem:a}
Let $T$ be a connected finite graph (possibly with multiple edges). Then either $T$ is an ADE diagram or $T$ contains an extended ADE, and not both. 
\end{lem}

\pf If $T$ does not contain an extended ADE, $\tilde{A}_n$ then no cycles, so a tree. not contain $\tilde{D}_n$ so then at most one branch point of valence $=3$. So its a $T_{pqr}$, 

% horizontal line with dots, p vertices left with meet, q in top, and r in right leg, all counting the connecting vertex. 

Assume that $p \leq q \leq r$. Not contain $\tilde{E}_6$ so that $p \leq 2$. Not contain $\tilde{E}_7$, so then $q \leq 3$. Not contain $\tilde{E}_8$ so that $r \leq 5$. But then $T_{1,1,n}$, $T_{2,2,n}$, or $T_{2,3,3}$, $T_{2,3,4}$, $T_{2,3,5}$, and these are $A_n$, $D_{n+2}$, $E_6$, $E_7$, or $E_8$, respectively. \qed \\


There are two `birthplaces' for extended ADE diagrams. The first is additive functions on graphs, the second is Tits quadratic forms of graphs. 


\begin{dfn}
Let $T$ be a finite connected graph on a vertex set $\{1,\ldots,n\}$. Then an additive function on $T$ is a function $a: \{1,\ldots,n\} \to \N_{>0}$ such that for every $i$
	\[
	\sum_{\text{there is edge i - j}} a_j = 2a_i
	\]
whee $a_i = a(i)$. It is subadditive if less than or equal to. Strictly subadditive if strict inequalty. 
\end{dfn}



Example
% two dots connected with 1s
is a subadditive function since $2 \geq 1$. Could there be an additive function? We would need $2a_1=a_2$ and $2a_2=a_1$, impossible. 



Example
% two vercied, double edge between
Now $a_1=1=a_2$ is an additive function because we count each edge separately. $2a_1=2a_2$.

Ecample
% three edges now
We would need $2a_1 \leq 3a_2$, $2a_2 \leq 3a_1$, impossible. 



Example 
$\tilde{D}_5$

% Draw double off two vertices label in rows abcdef
%f=1
%d=2
%b=1
%c=2
%a=1
%e=1
% go through logic.
So carries additive function.


The crucial observation is that if $T$ is the \mc graph of a subgroup of $\SL_2(\C)$ (replace each left/right arrow with dash), then labeling each vertex with the dimension of the corresponding representation is an additive function! This is because we tensor with the given 2-dimensional representation $\rho$, and connect $\rho_i$ to all the $\rho_j$ appearing in $\rho \otimes \rho_i$. So
	\[
	2\dim \rho_i = \dim(\rho \otimes \rho_i) = \sum_{\rho_i - \rho_j} \dim \rho_j.
	\]


\begin{thm}
A graph $T$ carries an additive function if and only if it is extended ADE. It carries a strictly subadditive function if and only if it is ADE. 
\end{thm}





\begin{lem} \label{lem:b}
The extended ADE graphs carry additive functions, and the ADE graphs carry strictly subadditive functions. 
\end{lem}

\pf Write them down. \qed \\


We need to reinterpret additive functions. Write a function $a: \{1,\ldots,n\} \to \N_{>0}$ as a column vector $a= \begin{pmatrix} a_1 \\ \vdots \\ a_n \end{pmatrix}$. Then $a$ is subadditive if and only if $2 \begin{pmatrix} a_1 \\ \vdots \\ a_n \end{pmatrix} \geq \begin{pmatrix} \sum a_j \\ \vdots \\ \sum a_j \end{pmatrix}$ % on right of each row 1-j, 2-j, \vdots n-j

$= \text{indicdence matrix} \cdot \begin{pmatrix} a_1 \\ \vdots \\ a_n \end{pmatrix}$, i.e. $2I- A$ has nonnegative entries, where $I$ is the identity matrix and $A$ is the indicence matrix and $[A]_{ij}$ is the number of edges between $i$ and $j$. 

% incidence D_4
%label 1, 4,3,2
%A = 0,0,0,1; 0,0,0,1; 0,0,0,1; 1,1,1,0
% subadditive 1,2,1,1

%A.a = column 2,2,2,3
%2a = 2,2,2,4 \geq 2,2,2,3= Aa

%$C=2I-A$ is called the Coxeter matrix of $T$.















\begin{lem} \label{lem:c}
If $T$ admits an additive function, then every subadditive function on $T$ is additive.
\end{lem}

\pf Set $C=2I_n - A$, where $A$ is the incidence matrix of $T$.. Assume $a$ is an additive function and $b$ is a subadditive function. We need show that $b$ is additive. Consider $b^TCa$. Since $a$ is additive, $Ca=0$. But we also know $b^TCa=b^TC^Ta$ since $C$ is symmetric, which is $(Cb) \cdot a$, which is a positive linear combination of entries of $Cb$ (since entries of $a$ are positive). But then $Cb=0$, which implies that $b$ is additive. \qed \\


\begin{cor} \label{cor:d}
Every subadditive function on an extended ADE diagram is additive. 
\end{cor}


\begin{lem} \label{lem:e}
Suppose that $T \subsetneq T'$ are finite connected graphs. If $T'$ carries a subadditive function $a$, the restriction of $a$ to $T$ is strictly subadditive. 
\end{lem}

\pf We know 
	\[
	2a_i \geq \sum_{i-j \in T'} a_j \geq \sum_{i-j \in T}
	\]
where second inequality follows since every edge in $T$ is an edge in $T'$. Since $T' \neq T$, there is at least one edge in $T'$ not in $T$, so the inequality must be strict. \qed \\




We can now prove a previously stated theorem.

\begin{thm}
A graph $T$ carries an additive function if and only if it is extended ADE. It carries a strictly subadditive function if and only if it is ADE. 
\end{thm}

\pf First, Lemma~\ref{lem:b} does $\leftarrow$ for both. For $\to$, assume carries an additive function $a$ and is not an extended ADE. Then by Lemma~\ref{lem:a}, either $T$ is ADE or $T$ \emph{strictly} contains an extended ADE. 

If $T$ is ADE, then $T$ carries a strictly subadditive function, contradicting Lemma~\ref{lem:c}. If $T$ strictly contains an extended ADE, then this ADE carries a strictly subadditive function by Lemma~\ref{lem:e}, contradicting Lemma~\ref{lem:c}. 

Finally if $T$ carries a strictly subadditive function and its not ADE, then $T$ contains an extended ADE, which must carry a strictly subadditive function by Lemma~\ref{lem:e}, contradicting Corollary~\ref{cor:d}. \qed \\


\begin{cor} \label{cor:point}
The \mc graph of a binary polyhedral group is extended ADE, with additive function given by the dimensions of the irreducible representations. In particular, the `extra vertex' has value 1. 
\end{cor}






\subsection{Another Aside: The Quadratic Form of a Graph}

\begin{dfn}
Let $T$ be a finite connected graph, possibly with multiple edges. A quadratic form of $T$, also known as a Tits form, is the polynomial $q_T(x_1,\ldots,x_n)= \sum_{i=1}^n x_i^2 - \sum_{i-j} x_ix_j$, where as usual we count edges with multiplicity. 
\end{dfn}


\begin{ex}
% 1 - 2
$q_T(x_1,x_2)= x_1^2+x_2^2-x_1x_2$.

% 1 triple line 2
$q_T(x_1,x_2)= x_1^2+x_2^2-3x_1x_2$
\end{ex}

Observe that $q(\mathbf{x})= \frac{1}{2}\mathbf{x}^TC\mathbf{x}$, where $C$ is the Coxeter matrix of $T$. So we wonder if $q_T$ is related to (sub)additive functions. 

\begin{thm}
The quadratic form $q_T$ is positive definite, i.e. $q(x) \geq 0$ for all $x$ and only zero for $x=0$, if and only if $T$ is an ADE diagram, and positive semidefinite, i.e. $q(x) \geq 0$, if and only if $T$ is extended ADE. 
\end{thm}

The theorem can be proved directly. Show that if $q_T$ is positive definite, then $T$ does not contain any cycles, or more than one branch point, or a vertex of degree $>3$. So $T$ is a $T_{pqr}$ tree. Show that $q_T$ is positive definite if and only if $1/p+1/q+1/r>1$., so then ADE. \qed \\

% 1 -2 -3 triangle cycle:
% x^2+y^2+z^2 - xy - xz - yz= 1/2( (x-y)^2+(x-z)^2+(y-z)^2)
% extended ADE but not ADE so should find zero.
% q(1,1,1)=0


% Thm -> q pos semidefinite - > q is sum of squares. 

One can also prove the theorem by translating $q_T$ into the additive function notation: 

\begin{prop}
If $a$ is an additive function on $T$, then $q_T$ is positive semidefinite. (also strictly subadditive then positive definite). 
\end{prop}

\pf Assume $a$ is an additive function. For each edge $e:i-j$, define $q_e(x_1,\ldots,x_n)= \dfrac{1}{2a_ia_j} (a_ix_j-a_jx_i)^2$. The coefficient of $x_i^2$ is $\dfrac{1}{2a_ia_j} a_j^2= \dfrac{a_j}{2a_i}$. The coefficient of $x_ix_j$ is $\dfrac{1}{a_ia_j} (-2a_ia_j)= -1$. Consider the sum $\sum_e q_e(x_1,\ldots,x_n)$. Coefficient of $x_ix_j$ is number of edges $i-j$. The coefficient of $x_i^2$ is $\sum_{\text{Edges e containing i}} \dfrac{a_j}{2a_i}= \dfrac{1}{2a_i} \sum a_j=1$. So $q_T= \sum q_e$'s is a sum of squares, so positive semidefinite. 













\section{Invariant Theory}


This is a transition from groups and graphs to commutative algebra and algebraic geometry. 


\begin{ex}
Consider $C_2 \subseteq \SL_2(\C)$, with generator $\two{-1}{0}{0}{-1}$. Then $C_2$ acts on $\C^2$ by $\sigma(p)= -p$ (or, rotates $p$ by $\pi$). The quotient space of this group action has for points the orbit of the action: for every nonzero point $\{p,-p\}$ (except for 0). A fundamental domain for this action, i.e. a subset of $\C^2$ containing exactly one point from each orbit. 

% axes with origin dotted and positive half of x-axis and same for positive y-axis, upper half plane and dark x-axis. arrow then get a cone. 
More precisely, we can define a continuous surjective map $\pi: \C^2 \to \text{cone}$ such that the fibers of $\pi$ are precisely the orbits of the action. Choose coordinates so that the cone is defined by $y^2=xz$, then $\pi(u,v)= (u^2,uv,v^2)$ works. 

To be more systematic, instead consider the ring of polynomials $\C[u,v]$, though of as the set of polynomials on $\C^2$. [The function $u$ picks out the first coordinate of a point $p \in \C^2$ and so forth.] 

The polynomial function on the quotient space $\C^2/C_2$ are exactly the polynomials on $\C^2$ that are constant on orbits. That is, $\{f \in \C[u,v] \colon f(p)=f(-p) \text{ for all } p \in \C^2\}=\C[u^2.uv,v^2] \subseteq \C[u,v]$, note $\C[u^2,uv,v^2] \cong \C[x,y,z]/(y^2-xz)$. Observations about the ring $\C[u^2,uv,v^2]$: commutative, graded, noetherian, domain, dimension 2, integrally closed, Cohen-Maccaulay (in fact, gorenstein), the polynomial ring $\C[u,v]$ is a finitely generated module over it, even reflexive, in fact $\C[u,v] \cong R \oplus (Ru + Rv)$, every indecomposable reflective $R$-module appears as a direct summand of polynomial ring $\C[u,v]$. 
\end{ex}


Now the question is how many of these are specific to this example? In fact, most are always true. 



\subsection{Classical invariant theory (of finite groups)}

Finite groups so no tori or Lie groups. 

For this material, we follow Kraft-Procesi. In particular, we take a coordinate free approach whenever possible. Now let $k$ be an infinite field and $W$ a finite dimensional vector space. We say that a function $f: W \to k$ is regular if polynomial in the elements of some basis of $W$. [Observe this is independent of basis.] Let $k[W]$ be the ring of regular functions on $W$. If $\{x_1,\ldots,x_n\}$ were a basis for $W^*=\Hom(W,k)$, then $k[W] \cong k[x_1,\ldots,x_n]$ is a polynomial ring. This is because the field is infinite---not true for finite fields (obstruction nonzero vanishing functions). 

A regular function $f \in k[W]$ is homogeneous of degree $d$ if $f(\lambda w)=\lambda^d w$ for all $\lambda \in k$ and $w \in W$. Concretely in terms of basis for $W^*$, this means that $f$ is a linear combination of monomials of degree $d$: $x_1^{d_1} \cdots x_n^{d_n}$ with $d_1+\cdots+d_n=d$. Since every polynomial is a sum of such monomials, every polynomial is a sum of homogeneous polynomials, i.e. $f \in k[W]$ is uniquely a sum of homogeneous polynomials, so $k[W] \cong \bigoplus_{d \geq 0} k[W]_d$, where $k[W]_d=$ homogeneous regular functions of degree $d$. In particular, $k[W]$ is a graded ring, i.e. $A= \bigoplus A_i$ as abelian groups such that $A_iA_j \subseteq A_{i+j}$. 


% k[W] ring of regular functions
% \cong k[x_1,\ldots,x_n]
% fix a basis $e_1,\ldots,e_n$ for $W$ then $x_1,\ldots,x_n$ are the dual basis for $W^*$, $x_i(e_j)=\delta_{ij}$.
% \bigoplus_d k[W]_d homogeneous space.


Suppose we have a subgroup $G \subseteq \GL(W)$, or more generally a representation $\rho: G \to \GL(W)$. This gives an action of $G$ on $W$: $gw=\rho(g)w)$. In turn, this gives an action of $G$ on $k[W]$, $(gf)(w)=f(g^{-1}w)$. The $(-1)$-power is needed to get a left action. Moreover, this action is compatible (in fact the same as) the action of $G$ on the dual space $W^*$. Keep in mind that $k[W]_1$ are the linear maps from $W \to k$, i.e. $W^*$. In fact, $k[W]_d= \sym_d(W^*)$, the d\tss{th} symmetric power of $W^*$, as such it inherits the action of $G$ on $W^*$. 

\begin{dfn}[Invariant]
A function $f \in k[W]$ is invariant ($G$-invariant) if $gf=f$ for all $g \in G$. Equivalently, $f(w)=f(gw)$ for all $g^{-1} \in G$, i.e. $g \in G$. We write $k[W]^G$ for the set $\{ f \in k[W] \colon f \text{ invariant}\}$.
\end{dfn}

One can check that $k[W]^G$ is a ring: each $g \in G$ acts as an automorphism of $k[W]$. 

\begin{ex}
Let $S_n$ be the symmetric group on $n$ letters. Now $S_n$ has an action on $W=k^n$ via $\sigma(e_i)=e_{\sigma(i)}$. Equivalently, $\sigma(a_1,\ldots,a_n)=(a_{\sigma^{-1}(1)}, \ldots, a_{\sigma^{-1}(n)})$. Then $S_n$ also acts on $k[W] \cong k[x_1,\ldots,x_n]$, where $\{x_i\}$ is the dual basis. What is $\sigma(x_i)$? We know $x_i(e_j)=\delta_{ij}$ so $(\sigma x_i)(e_j)=x_i(\sigma^{-1}(e_j))=x_ie_{\sigma^{-1}(j)}=\delta_{i \sigma^{-1}(j)}$. But this means $(\sigma x_i)(e_j)=\delta_{i \sigma^{-1}(j)}=1$ if and only if $\sigma^{-1}(j)=i$ if and only if $\sigma(i)=j$. Therefore, $\sigma x_i= x_{\sigma(i)}$. Generally for any $f \in k[x_1,\ldots,x_n]$, $(\sigma f)(x_1,\ldots,x_n)= f(x_{\sigma(1)},\ldots,x_{\sigma(n)})$. 

Now the question is which functions are invariant; that is, which functions of $k[x_1,\ldots,x_n]$ are independent of the order of the $x_i$? These are the symmetric functions, e.g. $x_1+\cdots+x_n$, $x_1\cdots x_n$, $x_1^7+\cdots+x_n^7$. 
\end{ex}


Elementary Symmetric Polynomials:
$s_1= x_1 + \cdots + x_n$
$s_2= x_1x_2+ x_1x_3+ \cdots + x_1x_n+x_2x_3+\cdots x_2x_n+x_3x_4+\cdots+x_{n-1}x_n$
$s_r= \sum_{i_1<i_2<\cdots<i_r} x_{i_1} x_{i_2} \cdots x_{i_r}$
$s_n= x_1\cdots x_n$


\begin{thm}[Fundamental Theorem of Symmetric Functions/Newton's Theorem]
Any symmetric polynomial in $x_1,\ldots,x_n$ (even over $\Z$) is uniquely polynomial in the elementary symmetric polynomials. 
\end{thm}

In particular, the $s_i$'s are algebraically independent of each other, i.e. there are no nontrivial polynomial relations among them. Then it must be that $k[x_1,\ldots,x_n]^{S_n}= k[s_1,\ldots,s_n]$, a polynomial ring (since the $s_i$ have no relations between them). 


\begin{ex}
$x^2+y^2 \in k[x,y]$ is symmetric. $s_1= x+y$, $s_2= xy$. So $x^2+y^2=(x+y)^2-2xy= s_1^2-2s_2$. 
\end{ex}


\begin{rem}
The power sums, $p_1=x_1+\cdots+x_n$, $p_2=x_1^2+\cdots+x_n^2$, $\ldots$, $p_n=x_1^n+\cdots+x_n^n$ also generate the ring of symmetric polynomials. The complete symmetric polynomials, the Schur polynomials, etc. all also generate the ring of symmetric polynomials. Hence, there are procedures from going from one set of these polynomials to another. The transition functions between them are crucial in the representation of $S_n$ and $\GL_n$ (Schur-Weyl Theory). 
\end{rem}


As another aside, the discriminant of $S_n$ acting on $x_1,\ldots,x_n$ is\footnote{This is often the square of what most refer to as the discriminant.}
	\[
	\Delta= \prod_{i<j} (x_i - x_j)^2
	\]
The discriminant is symmetric, so it must be a polynomial in $s_1,\ldots,s_n$. 

\begin{ex}
If $n=2$, then $\Delta=(x-y)^2=x^2-2xy+y^2= s_1^2- 4s_2$. If $n=3$, $\Delta=(x - y)^2(x-z)^2(y-z)^2= s_1^2s_2^2- 4s_2^3 - 4s_1^3s_3 - 27 s_3^2+ 18s_1s_2s_3$. 
\end{ex}

Given a polynomial $g(t) \in \C[t]$ with roots $a_1,\ldots,a_n$ (with multiplicity), the discriminant of $g$ is
	\[
	\Delta(g)= \Delta(a_1,\ldots,a_n)= \prod_{i<j} (a_i-a_j)^2
	\]

Observe $\Delta(g)=0$ if and only if $g$ has a repeated root. For example, $g(t)=t^2+bt+c$, then $\Delta(g)= b^2-4c$. An exercise for the reader is to show $\Delta(g)= (\det V)^2$, where $V$ is the Vandermonde matrix:
	\[
	V=
	\begin{pmatrix}
	1 & a_1 & a_1^2 & \cdots & a_1^{n-1} \\
	1 & a_2 & a_2^2 & \cdots & a_2^{n-1} \\
	\vdots & \vdots & & \ddots & \vdots \\
	1 & a_n & a_n^2 & \cdots & a_n^{n-1}
	\end{pmatrix}
	\]




\subsection{Restricting the Action of $S_n$}

We want to restrict the action of $S_n$ on $k[x_1,\ldots,x_n]$ to the subgroup $A_n \subseteq S_n$. All the symmetric functions are still invariant. Is anything else invariant? Notice that $(i\;j)\sqrt{\Delta}= - \sqrt{\Delta}$. So if $\sigma$ is an even permutation, $\sigma(\sqrt{\Delta})= \sqrt{\Delta}$.

FACT: $k[x_1,\ldots,x_n]^{A_n}= k[s_1,\ldots,s_n,\sqrt{\Delta}]$.


Indeed, we can think of $k[x_1,\ldots,x_n]^{A_n}$ as consisting of the symmetric polynomials and the sign-symmetric polynomials: $f(x_{\sigma(1)},\ldots,x_{\sigma(n)})= (-1)^{\sgn(\sigma)} f(x_1,\ldots,x_n)$. 


Moreover, $\Delta= (\sqrt{\Delta})^2$ is a polynomial in the `variables' $s_1,\ldots,s_n$. But then $k[x_1,\ldots,x_n]^{A_n}$ is isomorphic to a hypersurface ring: $k[y_1,\ldots,y_n,z]/(z^2-f(y_1,\ldots,y_n))$. 




First, a few basic questions about $k[W]^G$:

1. (Generators and Relations): Given a finite group $G \subseteq \GL(W)$, is the ring of invariants $k[W]^G$ a finitely generated $k$-algebra? 

2. If so, describe them explicitly and also is the ideal of relations among the generators finitely generated?If so, describe them explicitly.

Following theorem due to Hilbert and Noether:

\begin{thm}[First Fundamental Theorem of Invariant Theory for Finite Groups]
Let $k=\C$. The invariant ring $\C[W]^G$ is generated as a $\C$-algebra by at most $\binom{|G|+\dim W}{\dim W}$ homogeneous polynomials of degree at most $|G|$. 
\end{thm}


Note $\binom{n+d}{d}$ is the vector space dimension of homogenous polynomials of degree $d$ in $n$ variables. Hilbert proved finiteness as an application of the Hilbert-Basis Theorem [1890]. The proof given was  nonconstructive. There is a story that Gordan (rep. theory of binary forms, was constructive), is said to have a said thats not math thats theology. Mostly believed to be a story. Hilbert later gave a constructive proof. [1890s]. Noether gave the bound in the theorem which is tight, by showing $k[W]^G$ is generated by
	\[
	\left\{\dfrac{1}{|G|} \sum_{g \in G } gm \colon m \text{ runs over monomials of degree } \leq |G| \right\}
	\]


Sketch of Hilbert's (nonconstructive proof)

\begin{thm}[Hilbert Basis Theorem]
The polynomial ring $k[x_1,\ldots,x_n]$ is noetherian, i.e. every ideal of $k[x_1,\ldots,x_n]$ is finitely generated, where $k$ is a field.
\end{thm}

Let $S=k[x_1,\ldots,x_n]$, $R=k[x_1,\ldots,x_n]^G \subseteq S$. Let $I$ be the ideal of $R$ generated by all invariants of positive degree. 

Exercise: If $I$ is a finitely generated ideal of $R=k[f_1,\ldots,f_t]$, say $I=Rf_1+\cdots+Rf_t$, then $\{f_i\}$ generate the ring of invariants as a $k$-algebra. The proof follows by induction on the degree. 


We know that $IS$ is a finitely generated ideal of $S$ by the Hilbert Basis Theorem. Define the Reynold's operator 
	\[
	\begin{split}
	\rho&: S \to R \\
	f&\mapsto \dfrac{1}{|G|} \sum_{g \in G} gf.
	\end{split}
	\]
Observe that 

1. $\rho(S) \subseteq R$. We have seen this before in a different form.
2. $\rho$ fixes $R$ elementwise. 
3. $\rho$ is a ring homomorphism, and is $R$-linear: if $h \in S^G$, $f \in S$, then $\rho(hf)=h \rho(f)$
4. For any ideal $J$ of $R$, $JS=\{ \sum as \colon a \in J, s \in S\}$. So $\rho(JS)=\{ \rho(\sum as) \colon a \in J, s \in S\}= \{ \sum \rho(as) \colon a \in J, s \in S\}= \{ \sum \rho(s)a \colon a \in J, s \in S\}= \{ \sum ra \colon r \in R, a \in J\}= J$. But then $\rho(JS)=J$. (3rd $=$ sign follows from 3) and last from $\rho$ maps $S$ onto $R$. 


Then the ideal $I$ generated by the invariants of positive degree is the same as $\rho(IS)$, and $IS$ is finitely generated so $I$ is as well. 


% Chekov gun quote

\begin{thm}[The Second Fundamental Theorem of Invariant Theory for Finite Groups]
The invariant ring is finitely generated. [Hilbert's Syzygy Theorem]
\end{thm}


There are versions of the 1st and 2nd Fundamental Theorem of Invariant Theory for many classes of groups.











%\usepackage{tikz}
%\usetikzlibrary{calc,matrix,arrows,shapes.geometric,positioning,decorations.pathmorphing}
%\usepackage{tikz-cd}
%\usepackage{tkz-euclide}
%\usepackage{pgfplots}
%\pgfplotsset{compat=1.8}
%\usepgfplotslibrary{polar}
%\usepgflibrary{shapes.geometric}
%\usetikzlibrary{calc}
%\pgfplotsset{my style/.append style={axis x line=middle, axis y line=middle, xlabel={$x$}, ylabel={$y$}, axis equal}}
%\tikzset{snake it/.style={-stealth,
%decoration={snake, 
%    amplitude = .4mm,
%    segment length = 2mm,
%    post length=0.9mm},decorate}}



%\[
%\begin{tikzcd}
%\mathbb{Z} \arrow{r}{r} & \mathbb{Q} \arrow[yshift=1ex]{r}{g_1} \arrow[yshift=-1ex]{r}{g_2} & A
%\end{tikzcd}
%\]

%\[
%\begin{tikzcd}
%\varinjlim M_i \arrow[dotted,yshift=1ex]{rr}{\theta}& & \overline{\varinjlim M_i} \arrow[dotted,yshift=-1ex]{ll}{\overline{\theta}}\\
%& M_i \arrow{ul}{\alpha_i} \arrow[swap]{ur}{\beta_i} \arrow{d}{\varphi^i_j} & \\
%& M_j \arrow[bend left=50]{uul}{\alpha_j} \arrow[bend right=50,swap]{uur}{\beta_j} & 
%\end{tikzcd}
%\] 





%\[
%\begin{tikzcd}
%V \arrow[out=0,in=90,loop]{}
%\end{tikzcd}
%\]




